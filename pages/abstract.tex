\chapter{\abstractname}

\gls{vr} technology can benefit architectural design by providing an immersive and intuitive medium for collaboration, review and evaluation of the developed concepts.
The early prototypes were developed which indicated that users can be startled by certain unexpected events in immersive \gls{vr} environments.
To ensure satisfaction with the collaboration experience, I conducted a study on \gls{wa} during architectural collaboration in immersive \gls{vr} environments.
This paper presents the utility of using audio awareness in improving \gls{wa}.
I observed an overall positive effect of audio feedback on informing the users about changes happening in a digital shared workspace. Additionally, stereo headphones proved to be sufficient for delivering the auditory cues, especially, when combined with visual support.
In conclusion, awareness is only one of the factors that has to be taken into account when considering collaboration in immersive \glspl{ve}, along with interactions and user presentation.
This thesis supports previous research on audio awareness and shows its utility in the explored setting.

\begin{comment}
However, this immersiveness in combination with the inherently limited perceptual information available in \glspl{ve} creates a bizarre situation, where on one hand, the visual information is very convincing, but on the other, the familiar perceptual feedbacks from the environment and our own actions are stripped away.
Starting with the question of how this situation affects the satisfaction with collaboration experience, I arrive at the concept of \gls{wa}. I build upon the design of a previous study on \gls{wa} from the field of \gls{cscw} and adapt it into a study of architectural collaboration in an immersive \gls{vr} environment. 
The results of the conducted experiments support previous research and show the high benefit of using additional audio feedback to support \gls{wa}.
\end{comment}

