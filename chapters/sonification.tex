\chapter{Properties and Applications of Sound} % Monitoring with sound? Auditory displays? Auditory cues? Sound effects?
% TODO: if the chapter name remains Sonification, I need to give a definition, I guess

% The study of sound effects dates back to bbc-years book
The use of sound effects is not a new invention, already in 1931 certain sound effects were used for their evocative ability in bringing up strong memories, images, or feelings in the mind \cite{bbc_yearbook_1931}.

% Theory from earcons and icons, as the base building blocks for sound feedback

\paragraph[Sound properties]{} 
% 3D user interfaces: auditory cues, affordamces of different auditory displays, and there's sonification in the end
The auditory sensory system is the second-most used sensory channel after the visual channel \cite[p.~46]{jr_3d_2017}. As with the visual system, the auditory system provides us with different localization cues that allow us to determine the direction the sound is coming from and distance to it. The main types of auditory cues are:
\begin{description}
	\item[Binaural cues] - the direction of the sound source is derived through comparing the sound waves received at each individual ear. The limitation of binaural cues is that there are certain positions around the listener, where the cues are ambiguous and it is not possible to uniquely identify the direction.
	\item[Head-Related Transfer Functions (HRTFs)] - spatial filters that are applied to the original sound when it goes through the listener's torso, shoulders, head, and the outter ears. The latter contain different notches and grooves that elevate of suppress sound depending on which direction it came from. HRTFs help solve ambiguous cases caused by binaural cues.
	\item[Reverberation] - the collection reflected sound waves from the surfaces in the environment. Besides providing the listener with spacial information about the layout of the environment, reverberation allows them to determine the distance to the sound source.
	\item[Sound intensity] (loudness) - the primary cue in determining the sound source's distance. It is also a very simple cue to implement.
\end{description}
The authors also note that auditory combined with visual cues can be used to form better spatial perception of the environment. Additionally, the familiarity with the environment can influence the listener's ability to locate the sound source.

Another important aspect for presenting the aural information to the listener is the selection of an auditory display \cite[p.~153]{jr_3d_2017}. It defines in which way the audio is synthesized, how it is presented to the listener % TODO: do you mean the medium?
, and the ways in which the auditory cues are used in the application.
Different approaches to audio synthesis (i.e. using mathematical models, or sound sampling) have direct influence on the resulting auditory cues. For example, in case of taking HRTF measurements during sound sampling, this is usually done in echo-free environments, therefore, the recorded sound is stripped of the reverberation effect.
The authors point out several different ways in which audio displays can be used:
\begin{description}
	\item[localization] determining the direction and location of a sound;
	\item[sonification] turning information into sounds;
	\item[ambient effects] adding realism and the sense of immersion in a 3D application;
	\item[sensory substitution] playing a role of another perceptual cues (i.e. emitting a sound instead of haptic feedback); 
	\item[annotation and help] providing additional guidance in the environment (i.e. recording vocally indicating the correct way, or the auditory annotations of a previous user about an artifact in the shared environment).
\end{description}

\paragraph[Sonification for monitoring]{}
\cite{hermann_sonification_2011} argues that sonification has great application in monitoring. While this might seem similar to the localization task, the main difference here is that cognition (or comprehension) is required, not only the positional information of the sound source - users must understand what is the meaning of the emitted sound.
% Monitoring
One of the problems with auditory displays is that they are a temporary medium, that is - not always available for review. The authors argue that this is the exact reason that makes auditory display a perfect match for the monitoring task. It is often carried out as a secondary task in parallel with one or more primary tasks, and is based on paying attention to the temporally-related changes to the state of the system.

% Types of monitoring
Monitoring can be classified into two types: information pull (direct monitoring) and push. The former is the case, when monitoring is a primary task and intentionally attended by the user, the latter treats it as a secondary task, and therefore the information has to "pushed" in order to be attended to. The push category can be additionally substituted into simply peripheral and serendipitous-peripheral. The idea is that peripheral monitoring can be vital to the conduction of the primary task, or just convenient (or serendipitous).

% Modes of listening
Different listening modes can also be classified according to the push and pull analogy. In this case, hearing is a push activity, where sounds are forced into our attention span, and listening is a pull, where the listener is intentionally attending to the information being conveyed through sound \cite{hermann_sonification_2011}. Other classifications categorize listening through the attention to different properties of sound. In this way, everyday listening and casual listening are exemplified by authors as the modes, where attributes of the sound source are being attended: "big lorries, small children, plastic cups being dropped, glass bottles breaking", etc. In contrast, in musical or reduced listening, humans attend to the attributes of the sound (pitch, intensity, timbre, and so on). Semantic listening mode involves interpretation of the message encoded in the sound. In this way, the same message with different intonations, loudness, etc. can be interpreter equally.

% Some interesting systems (AR-Kola)
% ...

% Pitfalls - annoyence and stuff
There is a number of pitfalls when designing auditory displays, most of them can be classified into either \textit{intrusion and distraction, fatigue and annoyance}, or audience related problems.
% intrusion and distraction, fatigue and annoyance
The former mainly explores the trade-off between intrusiveness of the sound and the communicating enough information. \cite{hermann_sonification_2011} reviews the different ways that were employed in an attempt to tackle this problem. For example, sounds that fit to the working environment ecologically, like ticking clock in the office environment, can be less intrusive due to the fact that it corresponds to the listener's expectation of the environment. With time listener stops paying so much attention to the ticking sound, because its perceived importance drops. In such cases, the minimal intrusion can be caused by variable speed, timbre, or intensity of sound to draw listener's attention, while not obstructing the work too much.

%% Audience based: emotional associations, auesthetic and acoustic ecology (allowing users to choose the ecology), and comprehesibility and audibility.
There can be different audience related problems, \cite{hermann_sonification_2011} reviews three: \textit{emotive associations}, \textit{aesthetic and acoustic ecology}, and \textit{comprehensiveness and audibility}. Different users need different information from, and thus sonifications of the data. Emotive association describes the problem of presenting information that is too detailed to the parties that do not need it (i.e. sonification of the weather reports for meteorologists and average consumers should be approached differently). Aesthetic and acoustic ecology corresponds to the idea of combining such sounds in an auditory display that they are compatible in terms of frequency and intensity balance, correspond to the same theme, and take the cultural and personal differences of the audience into account. Comprehensiveness and audibility of auditory cues can be highly dependent on the social and cultural norms, and therefore have to be designed with the listener target group in mind.
% TODO: put sound ecologies in the possible future work, if more sounds were to be added they should be competible

\paragraph[Earcons and Auditory icons]{}
% Check the OneNote


\begin{comment}
%(really small chapter, explaining the possible ways to implement auditory cues [auditory icons, earcons, …])

% Reference list
%% The Sonification Handbook

%% Sandell Kramer's book review
Sonification - The citation "non-speech audio to convey or perceptualize data"
%% Blattner. Earcons and icons

%% ? 3D User Interfaces

%% ? Kronland-Martinet, Real-time perceptial simulation of moving..

%% BBC Year Book
"The Symbolic, Evocative [bringing strong memories, images, or feeling to mind] Effect; e.g. the churning rhythm record used to express the confusion of a charwoman's mind in "Intimate Snapshots "; the use of sea sounds between all scenes in "The Flowers are not for You to Pick," expressing the inevitability of disaster."
%% Sound Immersion in First-Person Shooter


% TODO: also cite Gutwin 2002: the pre-research and the findings (the limitations of audio awareness, and "How and why does audio improve awareness?")
% TODO: maybe cite Gutwin 2011, too. THere was something on sound being a good tool or wharevs

Previous chapters implicitly established that auditory cues could potentially be a great way for providing awareness about the workspace. In this chapter I review sonification and the means it provides us with that can be utilized for interpretation of certain events or data. Sonification is a concept akin to visualization. Both can be applied to data, and while the visualization will do it in a visual form, sonification will use "non-speech audio to convey or perceptualize data".

\section{Sound Properties}
% I mention timbre in Experiments chapter
% + harmonics & halftones (and how they allow to differentiate between sounds), fundamental frequency
% Occlusion in audio SDKs
% 

% TODO: explain them here, but use as an example in later chapters. Potentially, like the chapter before
% HRTFs and stuff? Can tie the stereo headphones here as a exapmle of why they suck.

	
\section{Auditory Icons}
\section{Earcons}
\section{Summary}

Bridge: In the next chapter we will combine the knowledge from the previous chapters to …

\end{comment}