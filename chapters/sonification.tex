\chapter{Sonification}
%(really small chapter, explaining the possible ways to implement auditory cues [auditory icons, earcons, …])

% Reference list
%% The Sonification Handbook

%% Sandell Kramer's book review
The citation "non-speech audio to convey or perceptualize data"
%% Blattner. Earcons and icons

%% ? 3D User Interfaces

%% ? Kronland-Martinet, Real-time perceptial simulation of moving..

%% BBC Year Book
"The Symbolic, Evocative [bringing strong memories, images, or feeling to mind] Effect; e.g. the churning rhythm record used to express the confusion of a charwoman's mind in "Intimate Snapshots "; the use of sea sounds between all scenes in "The Flowers are not for You to Pick," expressing the inevitability of disaster."
%% Sound Immersion in First-Person Shooter


% TODO: also cite Gutwin 2002: the pre-research and the findings (the limitations of audio awareness, and "How and why does audio improve awareness?")
% TODO: maybe cite Gutwin 2011, too. THere was something on sound being a good tool or wharevs

Previous chapters implicitly established that auditory cues could potentially be a great way for providing awareness about the workspace. In this chapter I review sonification and the means it provides us with that can be utilized for interpretation of certain events or data. Sonification is a concept akin to visualization. Both can be applied to a dataset, and while the visualization will do it in a visual form, sonification will use "non-speech audio to convey or perceptualize data".

\section{Sound Properties}
% I mention timbre in Experiments chapter
% + harmonics & halftones (and how they allow to differentiate between sounds), fundamental frequency
% Occlusion in audio SDKs
% 

% HRTFs and stuff? Can tie the stereo headphones here as a exapmle of why they suck.
	
\section{Auditory Icons}
\section{Earcons}
\section{Summary}

Bridge: In the next chapter we will combine the knowledge from the previous chapters to …