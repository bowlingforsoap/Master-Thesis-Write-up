\chapter{Sonification}
%(really small chapter, explaining the possible ways to implement auditory cues [auditory icons, earcons, …])

In this chapter I will review sonification and look into the different tools it provides us with, which can be used for creating auditory awareness presentations. I will call the sounds emitted by elements in a workspace - auditory cues. Sonification is a concept akin to visualization. Both can be applied to a dataset, and while the visualization will do it in form of images (possibly interactive), sonification will use "non-speech audio to convey or perceptualize data" [from Sonification wiki: Kramer, Gregory, ed. (1994). Auditory Display: Sonification, Audification, and Auditory Interfaces. -- couldn't verify the source].
	
\section{Auditory Icons}
\section{Earcons}
\section{Summary}

Bridge: In the next chapter we will combine the knowledge from the previous chapters to …