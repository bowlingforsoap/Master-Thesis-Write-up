% !TeX root = ../main.tex
% Add the above to each chapter to make compiling the PDF easier in some editors.

\chapter{Introduction}\label{chapter:introduction}
\section{Motivation: Virtual Reality (VR) is a great enhancement for creative thinking tool-set in architecture.}

\section{Problem Description:  VR applications imply certain design consideration to avoid cybersickness and communications are limited}
\ldots according to (Churchill 1998[see Greenwald research]), but we still need to communicate the intentions of collaborators in some way.

\section{Hypothesis}


\chapter{Literature Review}
\section{Creative Thinking in Architecture: Explore what actions/interactions architects perform during this process} % Maybe General to Specific would be my repetetive structure?
Process Analysis: The main goal of Creative Thinking in Architecture. Tasks that can be solved utilizing the creative thinking process. The steps used in the creative thinking process. Common actions/interactions for different tasks at each step.

\section{Human-Computer Interaction (HCI) and Evaluation: Research in what way are the new HCI systems are validated/evaluated}
\subsection{HCI: The goals of the HCI field. Different sub-domains of HCI (psychology, computer science). Examples of specific types of interactions HCI deals with: VR, AR, etc.}
\subsection{Solution validation/evaluation in HCI: methods, and principles.}

\section{Collaborative Virtual Environments: Find out what we are going to start building our implementation upon}
\subsection{Virtual Reality status quo: Virtual Reality development (and state-of-art). Advantages and Limitations.  Design obligations when developing VR applications. }

\subsection{Interactions in Virtual Environments}
\subsection{Collaborative Virtual Environments (CVEs) status quo: Development of CVEs from Churchill (1998) to Greenwald (2017). CVE design considerations (HCI aspect) and the state of art.}
\subsection{Implementation approaches: Networking and VR (Maybe, more appropriate in Design and Implementation)}
a. Networking: Overview of the ways to connect apps for real-time, low latencey communication
b. VR: Overview and comparison of existing platforms for creating VR experiences


\begin{comment}
Methodology: approach to solving the problem; chosen HCI methodology for the final evaluation - no idea
a. Chosen HCI evaluation methodology
\end{comment}


\chapter{Concept: Show how exactly and what we are going to approach}
\section{Target Group: target audience that the experience would be tunned to; the needs and requirements of this target audience.}

\section{Describe system components: Networking, PCs, Head-Mounted Displays, User Locality}

\section{Target Actions: Actions that we are going to implement; Scenarios for and applications of these actions}

\section{Explain how we are going to solve miscommunication problems in Collaborative Virtual Reality Environment}

\section{Evaluation HCI Methodology: The way we are going to collect data and analyze the results.}


\chapter{Design and Implementation}
\section{Networking: Approach to establish real-time, low-latency communication between two VR headsets; framework for transferring actions over the network}

\section{Interactions: Design and implementation of machine-agnostic interactions, which are broadcasted to users (local and remote) for processing}

\section{Show how the previous two points come together}


\chapter{Evaluation of Collaboration experience}
\section{Evaluation test: documentation on the developed test structure}

\section{Evaluation Results: un-biased results of the tests}


\chapter{Conclusions}
\section{A discussion of the analytical results and implications of these}

\section{A discussion of limitations and perspectives}


\chapter{Summary}
\section{Describe what was researched}

\section{Describe the tools used}

\section{Describe evaluation process}

\section{Address the results and conclusions}


\begin{comment}
See~\autoref{tab:sample}, \autoref{fig:sample-drawing}, \autoref{fig:sample-plot}, \autoref{fig:sample-listing}.

\begin{table}[htpb]
  \caption[Example table]{An example for a simple table.}\label{tab:sample}
  \centering
  \begin{tabular}{l l l l}
    \toprule
      A & B & C & D \\
    \midrule
      1 & 2 & 1 & 2 \\
      2 & 3 & 2 & 3 \\
    \bottomrule
  \end{tabular}
\end{table}

\begin{figure}[htpb]
  \centering
  % This should probably go into a file in figures/
  \begin{tikzpicture}[node distance=3cm]
    \node (R0) {$R_1$};
    \node (R1) [right of=R0] {$R_2$};
    \node (R2) [below of=R1] {$R_4$};
    \node (R3) [below of=R0] {$R_3$};
    \node (R4) [right of=R1] {$R_5$};

    \path[every node]
      (R0) edge (R1)
      (R0) edge (R3)
      (R3) edge (R2)
      (R2) edge (R1)
      (R1) edge (R4);
  \end{tikzpicture}
  \caption[Example drawing]{An example for a simple drawing.}\label{fig:sample-drawing}
\end{figure}

\begin{figure}[htpb]
  \centering

  \pgfplotstableset{col sep=&, row sep=\\}
  % This should probably go into a file in data/
  \pgfplotstableread{
    a & b    \\
    1 & 1000 \\
    2 & 1500 \\
    3 & 1600 \\
  }\exampleA
  \pgfplotstableread{
    a & b    \\
    1 & 1200 \\
    2 & 800 \\
    3 & 1400 \\
  }\exampleB
  % This should probably go into a file in figures/
  \begin{tikzpicture}
    \begin{axis}[
        ymin=0,
        legend style={legend pos=south east},
        grid,
        thick,
        ylabel=Y,
        xlabel=X
      ]
      \addplot table[x=a, y=b]{\exampleA};
      \addlegendentry{Example A};
      \addplot table[x=a, y=b]{\exampleB};
      \addlegendentry{Example B};
    \end{axis}
  \end{tikzpicture}
  \caption[Example plot]{An example for a simple plot.}\label{fig:sample-plot}
\end{figure}

\begin{figure}[htpb]
  \centering
  \begin{tabular}{c}
  \begin{lstlisting}[language=SQL]
    SELECT * FROM tbl WHERE tbl.str = "str"
  \end{lstlisting}
  \end{tabular}
  \caption[Example listing]{An example for a source code listing.}\label{fig:sample-listing}
\end{figure}
\end{comment}