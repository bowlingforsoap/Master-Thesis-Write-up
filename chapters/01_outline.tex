% !TeX root = ../main.tex
% Add the above to each chapter to make compiling the PDF easier in some editors.




\chapter{Introduction}

\section{Motivation}
\paragraph{About CDP [Gerhard]} \gls{cdp} is a multidisciplinary project at the \textit{Technical University of Munich} supported by the \textit{Chair of Architectural Informatics} from the \textit{Department of Architecture}, the \textit{Chair of Augmented Reality} from the \textit{Department of Informatics}, and the \itshape{Leibniz-Rechnenzentrum M{\"u}nchen}.
About \gls{vr} Sketching Component [Sofia]
Lead to \gls{cdp} \gls{vr} Collaboration component
\section{Problem Description}
\gls{cve}s [Greenwald, history of CVE]
Link to groupware [still Greenwald, but a bit under the history of CVE]
WA in groupware [Gutwin]
Hypothesis: Additional environmental cues improve WA in an immersive architectural CVE.
\section{Outline}
In Introduction I will write about the main motivation for this work, which is extending the CDP with an immersive collaborative virtual reality component, and the main problems that come along with it, specifically, user awareness of each other's actions. Then I would state my initial hypothesis about the additional cues being able to make the experience more "comfortable".

The Literature Review section of the thesis investigates human information processing and presents auditory cues as a great and underused tool, which could be utilized for signaling and monitoring [1]. Then I turn to the Situation Awareness (SA) in VR, and how it is a measure of a user being able to perceive, interpret, and project the things happening within the environment [2]. We will see that situation awareness has a subfield called Workspace Awareness (WA), which deals exactly with the awareness users have of each other in groupware systems [3,4]. We will also discover that WA deals mainly with monitoring and secondary tasks, and how "comfortable experience" can be described in terms of WA framework. Last part of Literature Review section turns to the field of sonification to explore how it can help with monitoring activities, and the different types and properties of sound that could be unitized [5].

Concept/Discussion. Here I will state the purpose of the project and explain how I am going to solve the problem of WA in the collaborative architectural environment. I will classify the system according to WA framework, and then state that we want to answer the what and where WA questions with auditory cues and (possibly visual cues in the form of a radar view). I will mention the research by Gutwin (2011) and say that I will try to adapt it to our use case. Then I will describe what is needed to do it (what we need to study, experiment with, etc.) and what is needed to get the answers to my research question - whether the additional auditory (and visual cues) help with WA.

Experiments. This section will one by one introduce the experiments made in this research: introduction (with statement of purpose), assumptions (limitations), apparatus and process, results and discussion. Here we will have all pilots in 1 subsection, and all the proper experiments (sound spatial judgement and, possibly, WA study) will have their own subsection.

% Might be the same as summary, except for the future work
In the Conclusions section I will aggregate the all findings and prior discussion to try and give an answer to the research question. I will mention the limitations of the conducted study. This section will end with a look into the future work. 

In Summary I will go over the whole work step by step, recall what has been done, mention the results of the experiments

%\chapter{Collaborative Virtual Environments}
\chapter{Literature review}
\section{Human Information Processing}
\section{Workspace Awareness}
\section{Sonification}
\chapter{Concept}
\chapter{Experiments}
\section{Pilot studies}
\section{Spatial sound judgment study}
\section{Workspace awareness study}
\chapter{Conclusions}
\chapter{Summary}

%TODO
Acknowledgements 
Bibliography






%---------------Reference commands and structires

% \chapter{Introduction}\label{chapter:introduction}
% \section{Motivation: Teamwork is important. Creative thinking implies generation of ideas, collaboration, and communication. Virtual Reality (VR) is a great enhancement for creative thinking tool-set in architecture.}
%\subsection{Solution validation/evaluation in HCI: methods, and principles.}
\begin{comment}
Methodology: approach to solving the problem; chosen HCI methodology for the final evaluation - no idea
a. Chosen HCI evaluation methodology
\end{comment}


\begin{comment}
See~\autoref{tab:sample}, \autoref{fig:sample-drawing}, \autoref{fig:sample-plot}, \autoref{fig:sample-listing}.
\begin{table}[htpb]
  \caption[Example table]{An example for a simple table.}\label{tab:sample}
  \centering
  \begin{tabular}{l l l l}
    \toprule
      A & B & C & D \\
    \midrule
      1 & 2 & 1 & 2 \\
      2 & 3 & 2 & 3 \\
    \bottomrule
  \end{tabular}
\end{table}

\begin{figure}[htpb]
  \centering
  % This should probably go into a file in figures/
  \begin{tikzpicture}[node distance=3cm]
    \node (R0) {$R_1$};
    \node (R1) [right of=R0] {$R_2$};
    \node (R2) [below of=R1] {$R_4$};
    \node (R3) [below of=R0] {$R_3$};
    \node (R4) [right of=R1] {$R_5$};

    \path[every node]
      (R0) edge (R1)
      (R0) edge (R3)
      (R3) edge (R2)
      (R2) edge (R1)
      (R1) edge (R4);
  \end{tikzpicture}
  \caption[Example drawing]{An example for a simple drawing.}\label{fig:sample-drawing}
\end{figure}

\begin{figure}[htpb]
  \centering

  \pgfplotstableset{col sep=&, row sep=\\}
  % This should probably go into a file in data/
  \pgfplotstableread{
    a & b    \\
    1 & 1000 \\
    2 & 1500 \\
    3 & 1600 \\
  }\exampleA
  \pgfplotstableread{
    a & b    \\
    1 & 1200 \\
    2 & 800 \\
    3 & 1400 \\
  }\exampleB
  % This should probably go into a file in figures/
  \begin{tikzpicture}
    \begin{axis}[
        ymin=0,
        legend style={legend pos=south east},
        grid,
        thick,
        ylabel=Y,
        xlabel=X
      ]
      \addplot table[x=a, y=b]{\exampleA};
      \addlegendentry{Example A};
      \addplot table[x=a, y=b]{\exampleB};
      \addlegendentry{Example B};
    \end{axis}
  \end{tikzpicture}
  \caption[Example plot]{An example for a simple plot.}\label{fig:sample-plot}
\end{figure}

\begin{figure}[htpb]
  \centering
  \begin{tabular}{c}
  \begin{lstlisting}[language=SQL]
    SELECT * FROM tbl WHERE tbl.str = "str"
  \end{lstlisting}
  \end{tabular}
  \caption[Example listing]{An example for a source code listing.}\label{fig:sample-listing}
\end{figure}
\end{comment}