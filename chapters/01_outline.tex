% !TeX root = ../main.tex
% Add the above to each chapter to make compiling the PDF easier in some editors.


\chapter{Introduction}
\begin{comment}
Chapter plan:
-Context: introduce CDP, explain the need for the VR collaboration component
-Problem description: explain that virtual working environment provides only limited information from the workspace (citing works on Workspace Awareness). Example/Motivation: giant buildings moving at and through you

-Research question??

-Hypothesis: hypothesise that additional auditory cues promote workspace awareness in collaborative VR
-Thesis Overview: provide a high-level overview of the contents of this thesis (Chapter 1 is about.., Chapter 2 is about..)
\end{comment}

% Context
% About CDP Gerhard
\gls{cdp} is a multidisciplinary project at the Technical University of Munich developed by the Chair of Architectural Informatics from the Department of Architecture, the Chair of Augmented Reality from the Department of Informatics, and the Leibniz-Rechnenzentrum M{\"u}nchen.
%About \gls{vr} Sketching Component Sofia
At its core, \gls{cdp} is a multi-touch table \cite[p.~5]{lampe_cdp//vr-sketching_2017} that provides a lot of  functionality useful in urban planning and design (i.e. the ability to interactively add new buildings at a given site on the map, sketch on the \gls{vb}s, run solar envelope and wind simulations, etc.). One of the latest  development directions for the table is a \gls{vr} component, which aims to provide the existing 3D sketching capabilities in an immersive \gls{vr} environment. With the addition of the this component, comes a question of implementing a \gls{cve} to allow multiple people to use it simultaniously.


%Motivation
%Lead to \gls{cdp} \gls{vr} Collaboration component
 In this \gls{cve} users would have to interact with real-sized \gls{vb}s, which points out the initial concern and the motivation for this work. Imagine 2 users performing their separate tasks in the same \gls{cve}: while User 1 is occupied with sketching on the \gls{vb}s and going through the new ideas for an urban district, User 2 is busy with positioning the buildings in the environment to be more esthetically pleasing. The main question was - how would the moving buildings be perceived by a user that is ocuppied with their own task (for example, in case when a \gls{vb} moves through User 1), and how would this influence the quality of collaboration and the satisfaction with the experience in general.  
 

 %Problem description
 This brings up a problem of awareness different users have of each other in the system. \cite{gutwin_descriptive_2002} argues that "... workspace awareness is much harder to maintain in groupware workspaces than in face-to-face environments, and it is often difficult or impossible to determine who else is in the workspace, where they are working, and what they are doing". Authors point out that this is due to certain properties that any virtual workspace, and collaborative software (groupware) in particular have. This is mostly due to the fact that these systems provide only limited information about their current state to the user \cite[p.~414-415]{gutwin_descriptive_2002}. This comes as no surprise, because it would be virtually impossible and impractical to implement all the intricate details of the real world in software that tries to automate it. 
  

 %Hypothesis 
 To overcome these limitations, a user must be provided with the essential information about the environment. In this project, I explore what effect additional auditory cues have on the \gls{wa} of the collaborators. This work extends the \gls{wa} study presented in \cite{gutwin_chalk_2011} by putting ephasis on \gls{wa} in an immersive \gls{vr} environment.
 

% Overview
\paragraph{}
In the Literature Review chapter of this thesis I will explore how we as humans process infomation from the outside world, different channels available for this information. I will also introduce auditory cues as a useful tool for signling and monitoring. Then I turn to \gls{sa}, which can be shortly described as an up-to-date understanding of your current environment. Lastly, we will take a look at subfiled of \gls{sa} - \gls{wa}, and familiarize ourselves with \gls{wa} framework and the \gls{wa} study done by Gutwin et al. in \cite{gutwin_chalk_2011}. The \gls{wa} in Immersive Virtual Reality chapter will present the main user study conducted for this thesis and its results. In the summary ...


% Prev. Overview
\begin{comment}
In Introduction I will write about the main motivation for this work, which is extending the CDP with an immersive collaborative virtual reality component, and the main problems that come along with it, specifically, user awareness of each other's actions. Then I would state my initial hypothesis about the additional cues being able to make the experience more "comfortable".

The Literature Review section of the thesis investigates human information processing and presents auditory cues as a useful tool for signaling and monitoring [1]. Then I turn to the Situation Awareness (SA) in VR, and how it is a measure of a user being able to perceive, interpret, and project the things happening within the environment [2]. We will see that situation awareness has a subfield called Workspace Awareness (WA), which deals exactly with the awareness users have of each other in groupware systems [3,4]. We will also discover that WA deals mainly with monitoring and secondary tasks, and how "comfortable experience" can be described in terms of WA framework. Last part of Literature Review section turns to the field of sonification to explore how it can help with monitoring activities, and the different types and properties of sound that could be unitized [5].

Concept/Discussion. Here I will state the purpose of the project and explain how I am going to solve the problem of WA in the collaborative architectural environment. I will classify the system according to WA framework, and then state that we want to answer the what and where WA questions with auditory cues and (possibly visual cues in the form of a radar view). I will mention the research by Gutwin (2011) and say that I will try to adapt it to our use case. Then I will describe what is needed to do it (what we need to study, experiment with, etc.) and what is needed to get the answers to my research question - whether the additional auditory (and visual cues) help with WA.

Experiments. This section will one by one introduce the experiments made in this research: introduction (with statement of purpose), assumptions (limitations), apparatus and process, results and discussion. Here we will have all pilots in 1 subsection, and all the proper experiments (sound spatial judgement and, possibly, WA study) will have their own subsection.

% Might be the same as summary, except for the future work
In the Conclusions section I will aggregate the all findings and prior discussion to try and give an answer to the research question. I will mention the limitations of the conducted study. This section will end with a look into the future work. 

In Summary I will go over the whole work step by step, recall what has been done, mention the results of the experiments
\end{comment}

%\chapter{Collaborative Virtual Environments}
\chapter{Literature review}
\section{Human Information Processing}
\section{Awareness}
\subsection{Situation Awareness}
\subsection{Workspace Awareness}
\section{Sonification}
\chapter{Concept}
\section{Related Work: Chlak sound}
\paragraph{Intro}
\paragraph{Gutwin vs our study}

\begin{table}[]
  \caption{Study comparison}
  \label{table:study_comp}
  \begin{tabular}{|l|l|l|}
  \hline
                             & Gutwin                & This study           \\ \hline
  Primary (distraction) task & Same as the secondary & Contextually related \\ \hline
  Working (sound) plane      & ?Vertical             & Horizonatal          \\ \hline
  Sound type                 & Auditory icons        & Auditory icons       \\ \hline
  Object of analysis         & Workspace awareness   & Workspace Awareness  \\ \hline
  \end{tabular}
\end{table}

\paragraph{Summary}
\chapter{Experiments}
\section{Pilot studies}
\section{Spatial sound judgment study}
\section{Workspace awareness study}

\paragraph{When will recruitment and data collection commence?}


Recruitment of participants and data collection will commence on August 15th, 2018. Data collection will begin on September 3rd, 2018.
\paragraph{When will data collection be completed?}


Data collection will be completed by September 15th, 2018.
\paragraph{7.	Brief description in lay terms of the aim of the project, and outline of the research questions that will be answered (approx. 200 words):}


In my system users are able.. [brief description of the system]

The aim of the system is to analyze \gls{wa} of an architectural \gls{vr} application in a collaborative setting with and without additional auditory cues. \newline

Research questions: Can \gls{wa} in an architectural \gls{vr} application be improved by providing additional spatialized auditory cues from the environment? \newline

The system simulates the presence of 2 users in the same collaborative virtual environment in an architectural context: one user is the participant and the other is simulated by scipting the movements of a \gls{vb}. The translations of \gls{vb}s can be followed by an auditory cue in a shape of recording of a sound of concrete moving on concrete (\textbf{\textit{github link?}}). The participant observes the environement through an \gls{hmd}, and manipulates it with the help of controllers. User can get additional cues from the environment from the radar view

\paragraph{8.	Brief description of the method.} Include a description of who the participants are, how the participants will be recruited, and what they will be asked to do and how the data will be used and stored (Note: if this research involves patient data or health information obtained from the Ministry of Health, DHBs etc please refer to the UOHEC(H) Minimal Risk Health Research  - Audit and Audit related studies ):-

\paragraph{Experimental Design}

The independent variables are 1) auditory cues from moving \gls{vb}s and 2) a visual Radar view (minimap) of the surrounding environment. Auditory cues have 2 levels: turned on and muted. The radar has 2 levels also: shown and hidden.
The dependent variables will be \gls{wa}-related measurements: speed of the reaction when "catching" a \gls{vb} (\gls{sa} Levels 1,2: Perception and Comprehension), and the correctness in specifying it's translation direction (i.e. away from you, towards you, parallel to the left, etc.; \gls{sa} Level 3: Projection).

I am using a within-subjects experimental design, which consists of 3 groups:
\begin{enumerate}
	\item Auditory cues turned on, Radar hidden.
	\item Auditory cues turned on, Radar shown.
	\item Auditory cues.
\end{enumerate}
All participants will be tested in each group, but the order in which they are presented will be randomized to satisfy parametric assumption of unbiased results by eliminate the learning effect. The experiment will take approximately 30 minutes in total.

\paragraph{Participants}
Participants will be recruited from the \gls{tum}. This can include undergraduate students, postgraduate students, support staff, academic staff and lecturers/professors. All participants wouldn't have experienced the system before to insure independent samples.

\paragraph{Recruitment}
(Isn't my target group - architects?) The recruitment will be done via promoting the project at the chair for Architectural Informatics from the faculty of Architecture at the \gls{tum}, as well as by inviting friends to take part. 

\paragraph{Reward}
Participants will be thanked and offered a chocolate bar at the conclusion of their session.

\paragraph{Estimated Number of participants}
I estimate requiring 12 or more participants, with the number of participants being a multiple of 6. This is based on the previous study by \cite{gutwin_chalk_2011} and the number of possible permutations of the 3 groups in my experiment design ($P_{3}=6$).


\paragraph{Participant Inclusion and exclusion criteria:}
\begin{enumerate}
	\item Participants must have normal, or corrected to normal vision
	\item Participants must speak English in order to answer questionnaires 
	\item Participants must have no upper limb impairments to either limb (left or right arm)
	\item Participants must have no impairment to either hand which affects their use of the hand/fingers or has pain in using either hand/fingers.
	\item Participants must be comfortable wearing and using a head-mounted display device and stereo headphones. 
\end{enumerate}

\paragraph{Questionnaires and Measures} \mbox{} \newline \newline
\textit{Self-report Questionnaires} \newline
\begin{enumerate}
	\item Simulator Sickness questionnaire (Virtual Reality sickness or Cybersickness?)
	\item SBSOD, short for Santa Barbara Sense of Direction (Hegarty et al. 2002), is a self-report measure for measuring people’s judgments about environmental spatial abilities \cite{jr_3d_2017}  \newline
\end{enumerate}

\textit{Performance} \newline
The system will collect the following data:
\begin{enumerate}
	\item The fact of "catching" the \gls{vb}.
	\item The time it took for participant to catch the \gls{vb}, counting from the initiation of it's tranlation.
	\item \gls{vb} translation direction guess.
\end{enumerate}


\paragraph{Procedure}

Data will be collected from the participants from the questionnaires as well information collected via the system. The experiment will take approximately 30 minutes. Before the experiment begins, all participants are provided with information sheets and consent forms, which explains the experiment in detail. If they agreed and signed the consent forms, they proceed to the experiment. The participants will begin by filling out the Demographic Survey. They will then complete the Hand Visualization Realism Questionnaire which has participants evaluate the 4 hand visualizations in terms of realism. This is done to replicate the previously described study and allow for our findings to factor in realism of the hand visualizations.
Before the session, the participants will be explained the system set up and be demonstrated how it works. Instructions will be given to ensure the system is operated with a flat hand or index finger on the desk/table instead of having the hand in mid-air. Participants will be randomly assigned one of the four hand visualizations. Participants will be asked to play a “warm up” game with their assigned hand visualization. The “warm up” game has a similar tile board to TheraMem, however, has a different task. The task is to use the hand visualization to move the visualized hand/finger to a colour identified tile (repeat for 10 tiles). The purpose of the warm up task is to get the participant comfortable with the game interaction without having them practice the actual TheraMem game. The participant will play the warm up game twice (both mirrored and non-mirrored).
Once participants understand how to play the game, the procedure for rounds of the game will be explained. The order of the mirroring conditions will be randomly assigned. The participant will then play a round of TheraMem with their assigned hand visualization and given starting mirroring condition. Each round of the game will take approximately 2-4 minutes. After each round of the game, they will be asked to answer 16 Session Self-Report Sheet questions. The previous participant answers will be displayed to the participant so that they will be able how they answered questions from previous round to give a relative (differential) judgement. They will play a total of two rounds of TheraMem in the experiment (one for each mirroring condition).
At the conclusion of the experiment, participants will be informed of the actual difference between the conditions and will be asked not to reveal this to peers for the immediate future. They will be asked if they would like to receive the results of the study and asked about their interest in participating in future studies. All participants will be thanked and given a chocolate bar. 

\paragraph{Data Storage}

The results of the project may be published and will be available in the Technical University of Munich Library (Munich, Germany) but every attempt will be made to preserve anonymity.

\paragraph{Disclose and discuss any potential problems and how they will be managed}
Generally, I do not anticipate any problems arising out of this experiment for participants. However, there might be some fears about the confidentiality and anonymity of participants. Regarding this, participants will be informed that their participation is voluntary and that all their details will be kept confidential and anonymous. No sensitive information will be collected. Each participant will be provided with an information sheet and a consent form to read explaining this before starting the experiment. 
Experiencing virtual reality and wearing a head-mounted display can cause unintended temporary side effects. Rare temporary side effects of experiencing virtual reality can include motion sickness, eye discomfort and fatigue. These side effects are temporary and there are no reported cases of them being permanent. We don’t foresee these side effects occurring; however, we will screen participants if they are predisposed to motion sickness or have had any negative experiences in virtual reality before. If they are predisposed to motion sickness or have had negative experiences using virtual reality in the past, they will be warned about the possibility of experiencing those again. I have taken steps to mitigate potential motion sickness by using modern hardware (HTC Vive HMD), which have low latency and high frame update rate. However, if these rare side effects do occur, the participant will stop the experiment for the moment and take a break. They will be informed that they can stop the experiment if they wish or they can continue if the side effect has alleviated. 
It is possible that the outcome of this project will be used by other reports, publications or conferences. In this situation, no individuals will be identified. 


\chapter{Conclusions}
\section{Study limitations}
% the fact that we only sample the level 1 SA/WA (we won't be going into sampling direction of translation guesses from the participants)
\chapter{Summary}

%TODO
Acknowledgements 
Bibliography






%---------------Reference commands and structires

% \chapter{Introduction}\label{chapter:introduction}
% \section{Motivation: Teamwork is important. Creative thinking implies generation of ideas, collaboration, and communication. Virtual Reality (VR) is a great enhancement for creative thinking tool-set in architecture.}
%\subsection{Solution validation/evaluation in HCI: methods, and principles.}
\begin{comment}
Methodology: approach to solving the problem; chosen HCI methodology for the final evaluation - no idea
a. Chosen HCI evaluation methodology
\end{comment}


\begin{comment}
See~\autoref{tab:sample}, \autoref{fig:sample-drawing}, \autoref{fig:sample-plot}, \autoref{fig:sample-listing}.
\begin{table}[htpb]
  \caption[Example table]{An example for a simple table.}\label{tab:sample}
  \centering
  \begin{tabular}{l l l l}
    \toprule
      A & B & C & D \\
    \midrule
      1 & 2 & 1 & 2 \\
      2 & 3 & 2 & 3 \\
    \bottomrule
  \end{tabular}
\end{table}

\begin{figure}[htpb]
  \centering
  % This should probably go into a file in figures/
  \begin{tikzpicture}[node distance=3cm]
    \node (R0) {$R_1$};
    \node (R1) [right of=R0] {$R_2$};
    \node (R2) [below of=R1] {$R_4$};
    \node (R3) [below of=R0] {$R_3$};
    \node (R4) [right of=R1] {$R_5$};

    \path[every node]
      (R0) edge (R1)
      (R0) edge (R3)
      (R3) edge (R2)
      (R2) edge (R1)
      (R1) edge (R4);
  \end{tikzpicture}
  \caption[Example drawing]{An example for a simple drawing.}\label{fig:sample-drawing}
\end{figure}

\begin{figure}[htpb]
  \centering

  \pgfplotstableset{col sep=&, row sep=\\}
  % This should probably go into a file in data/
  \pgfplotstableread{
    a & b    \\
    1 & 1000 \\
    2 & 1500 \\
    3 & 1600 \\
  }\exampleA
  \pgfplotstableread{
    a & b    \\
    1 & 1200 \\
    2 & 800 \\
    3 & 1400 \\
  }\exampleB
  % This should probably go into a file in figures/
  \begin{tikzpicture}
    \begin{axis}[
        ymin=0,
        legend style={legend pos=south east},
        grid,
        thick,
        ylabel=Y,
        xlabel=X
      ]
      \addplot table[x=a, y=b]{\exampleA};
      \addlegendentry{Example A};
      \addplot table[x=a, y=b]{\exampleB};
      \addlegendentry{Example B};
    \end{axis}
  \end{tikzpicture}
  \caption[Example plot]{An example for a simple plot.}\label{fig:sample-plot}
\end{figure}

\begin{figure}[htpb]
  \centering
  \begin{tabular}{c}
  \begin{lstlisting}[language=SQL]
    SELECT * FROM tbl WHERE tbl.str = "str"
  \end{lstlisting}
  \end{tabular}
  \caption[Example listing]{An example for a source code listing.}\label{fig:sample-listing}
\end{figure}
\end{comment}