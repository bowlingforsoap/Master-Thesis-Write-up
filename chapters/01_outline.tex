% !TeX root = ../main.tex
% Add the above to each chapter to make compiling the PDF easier in some editors.




\chapter{Introduction}

\section{Motivation}
\paragraph{About CDP [Gerhard]} \gls{cdp} is a multidisciplinary project at the \textit{Technical University of Munich} supported by the \textit{Chair of Architectural Informatics} from the \textit{Department of Architecture}, the \textit{Chair of Augmented Reality} from the \textit{Department of Informatics}, and the \textit{Leibniz-Rechnenzentrum M{\"u}nchen}.

\paragraph{About \gls{vr} Sketching Component [Sofia]} At its core \gls{cdp} is a multi-touch table \cite[p.~5]{lampe_cdp//vr-sketching_2017}. Over the years, a lot of new functionality was added to the project, some of which in form of plugins, i.e. solar envelope and wind simulation. Other functionality, while still relevant and finds its origins in \gls{cdp}, is not strictly connected to the main project. \cite{lampe_cdp//vr-sketching_2017} is an example of of a promising side development, which in the nearest future might become a proper part of \gls{cdp}. It aims to provide the 3D sketching capabilities present in the \gls{cdp} table in an immersive \gls{vr} application

\paragraph{Lead to \gls{cdp} \gls{vr} Collaboration component} 

\section{Problem Description}
\gls{cve}s [Greenwald, history of \gls{cve}]
Link to groupware [still Greenwald, but a bit under the history of \gls{cve}]
\gls{wa} in groupware [Gutwin]
Hypothesis: Additional environmental cues improve WA in an immersive architectural CVE.
\section{Outline}
In Introduction I will write about the main motivation for this work, which is extending the CDP with an immersive collaborative virtual reality component, and the main problems that come along with it, specifically, user awareness of each other's actions. Then I would state my initial hypothesis about the additional cues being able to make the experience more "comfortable".

The Literature Review section of the thesis investigates human information processing and presents auditory cues as a great and underused tool, which could be utilized for signaling and monitoring [1]. Then I turn to the Situation Awareness (SA) in VR, and how it is a measure of a user being able to perceive, interpret, and project the things happening within the environment [2]. We will see that situation awareness has a subfield called Workspace Awareness (WA), which deals exactly with the awareness users have of each other in groupware systems [3,4]. We will also discover that WA deals mainly with monitoring and secondary tasks, and how "comfortable experience" can be described in terms of WA framework. Last part of Literature Review section turns to the field of sonification to explore how it can help with monitoring activities, and the different types and properties of sound that could be unitized [5].

Concept/Discussion. Here I will state the purpose of the project and explain how I am going to solve the problem of WA in the collaborative architectural environment. I will classify the system according to WA framework, and then state that we want to answer the what and where WA questions with auditory cues and (possibly visual cues in the form of a radar view). I will mention the research by Gutwin (2011) and say that I will try to adapt it to our use case. Then I will describe what is needed to do it (what we need to study, experiment with, etc.) and what is needed to get the answers to my research question - whether the additional auditory (and visual cues) help with WA.

Experiments. This section will one by one introduce the experiments made in this research: introduction (with statement of purpose), assumptions (limitations), apparatus and process, results and discussion. Here we will have all pilots in 1 subsection, and all the proper experiments (sound spatial judgement and, possibly, WA study) will have their own subsection.

% Might be the same as summary, except for the future work
In the Conclusions section I will aggregate the all findings and prior discussion to try and give an answer to the research question. I will mention the limitations of the conducted study. This section will end with a look into the future work. 

In Summary I will go over the whole work step by step, recall what has been done, mention the results of the experiments

%\chapter{Collaborative Virtual Environments}
\chapter{Literature review}
\section{Human Information Processing}
\section{Awareness}
\subsection{Situation Awareness}
\subsection{Workspace Awareness}
\section{Sonification}
\chapter{Concept}
\chapter{Experiments}
\section{Pilot studies}
\section{Spatial sound judgment study}
\section{Workspace awareness study}

\paragraph{When will recruitment and data collection commence?}


Recruitment of participants and data collection will commence on August 15th, 2018. Data collection will begin on September 3rd, 2018.
\paragraph{When will data collection be completed?}


Data collection will be completed by September 15th, 2018.
\paragraph{7.	Brief description in lay terms of the aim of the project, and outline of the research questions that will be answered (approx. 200 words):}


In my system users are able.. [brief description of the system]

Aim. Analyze \gls{wa} of an architectural \gls{vr} application in a collaborative setting.

Research questions. Can \gls{wa} in an architectural \gls{vr} application be improved by providing additional spatialized auditory cues from the environment?
\paragraph{8.	Brief description of the method.} Include a description of who the participants are, how the participants will be recruited, and what they will be asked to do and how the data will be used and stored (Note: if this research involves patient data or health information obtained from the Ministry of Health, DHBs etc please refer to the UOHEC(H) Minimal Risk Health Research  - Audit and Audit related studies ):-

\paragraph{Experimental Design}

The independent variables are 1) auditory cues from moving \gls{vb}s and 2) a visual Radar view (minimap) of the surrounding environment. Auditory cues have 2 levels: turned on and muted. The radar has 2 levels also: shown and hidden.
The dependent variables will be \gls{wa}-related measurements: speed of the reaction when "catching" a \gls{vb} (\gls{sa} Levels 1,2: Perception and Comprehension), and the correctness in specifying it's translation direction (i.e. away from you, towards you, parallel to the left, etc.; \gls{sa} Level 3: Projection).

I am using a within-subjects experimental design, which consists of 3 groups:
\begin{enumerate}
	\item Auditory cues turned on, Radar hidden.
	\item Auditory cues turned on, Radar shown.
	\item Auditory cues.
\end{enumerate}
All participants will be tested in each group, but the order in which they are presented will be randomized to satisfy parametric assumption of unbiased results by eliminate the learning effect. The experiment will take approximately 30 minutes in total.

\paragraph{Participants}
Participants will be recruited from the \gls{tum}. This can include undergraduate students, postgraduate students, support staff, academic staff and lecturers/professors. All participants wouldn't have experienced the system before to insure independent samples.

\paragraph{Recruitment}
(Isn't my target group - architects?) The recruitment will be done via promoting the project at the chair for Architectural Informatics from the faculty of Architecture at the \gls{tum}, as well as by inviting friends to take part. 

\paragraph{Reward}
Participants will be thanked and offered a chocolate bar at the conclusion of their session. 

\paragraph{Estimated Number of participants}
I estimate requiring 12 or more participants, with the number of participants being a multiple of 6. This is based on the previous study by \cite{gutwin_chalk_2011} and the number of possible permutations of the 3 groups in our experiment design ($P_{3}=6$).


\paragraph{Participant Inclusion and exclusion criteria: }
\begin{enumerate}
	\item Participants must have normal, or corrected to normal vision
	\item Participants must speak English in order to answer questionnaires 
	\item Participants must have no upper limb impairments to either limb (left or right arm)
	\item Participants must have no impairment to either hand which affects their use of the hand/fingers or has pain in using either hand/fingers.
	\item Participants must be comfortable wearing and using a head-mounted display device and stereo headphones. 
\end{enumerate}

\paragraph{Questionnaires and Measures} \mbox{} \newline \newline
\textit{Self-report Questionnaires} \newline
 Our questions were adapted from the study we are replicating by Lin and Jorg4 (2016) and Argelaguet et al (2016).
Demographics and background: this section covers participant’s demographics (gender, age, and ethnicity) and experience with virtual/augmented reality
Hand Visualization Realism Questionnaire: Participants will evaluate how they perceive realism of the 4 hand visualizations participants could be assigned to interact with.
Q1-8 are adapted from the Lin and Jorg (2016)4 questionnaire 
Q9-16 are adapted from Argelaguet et al (2016)6 questionnaire \newline \newline
\textit{TheraMem Performance} \newline
The TheraMem game will provide evaluation measures based on their performance in the game.
Completion Time (Time taken to find all matching pairs of food images)
Number of Attempts (Tries taken to find all matching pairs of food images)


\paragraph{Procedure}

Data will be collected from the participants from the questionnaires as well information collected via the system. The experiment will take approximately 30 minutes. Before the experiment begins, all participants are provided with information sheets and consent forms, which explains the experiment in detail. If they agreed and signed the consent forms, they proceed to the experiment. The participants will begin by filling out the Demographic Survey. They will then complete the Hand Visualization Realism Questionnaire which has participants evaluate the 4 hand visualizations in terms of realism. This is done to replicate the previously described study and allow for our findings to factor in realism of the hand visualizations.
Before the session, the participants will be explained the system set up and be demonstrated how it works. Instructions will be given to ensure the system is operated with a flat hand or index finger on the desk/table instead of having the hand in mid-air. Participants will be randomly assigned one of the four hand visualizations. Participants will be asked to play a “warm up” game with their assigned hand visualization. The “warm up” game has a similar tile board to TheraMem, however, has a different task. The task is to use the hand visualization to move the visualized hand/finger to a colour identified tile (repeat for 10 tiles). The purpose of the warm up task is to get the participant comfortable with the game interaction without having them practice the actual TheraMem game. The participant will play the warm up game twice (both mirrored and non-mirrored).
Once participants understand how to play the game, the procedure for rounds of the game will be explained. The order of the mirroring conditions will be randomly assigned. The participant will then play a round of TheraMem with their assigned hand visualization and given starting mirroring condition. Each round of the game will take approximately 2-4 minutes. After each round of the game, they will be asked to answer 16 Session Self-Report Sheet questions. The previous participant answers will be displayed to the participant so that they will be able how they answered questions from previous round to give a relative (differential) judgement. They will play a total of two rounds of TheraMem in the experiment (one for each mirroring condition).
At the conclusion of the experiment, participants will be informed of the actual difference between the conditions and will be asked not to reveal this to peers for the immediate future. They will be asked if they would like to receive the results of the study and asked about their interest in participating in future studies. All participants will be thanked and given a chocolate bar. 

Data Storage

The collected data will be securely stored in a locked cabinet that only members apart of the research team (mentioned at the top) will be able to gain access to. Data obtained as a result of the research will be retained for at least 5 years in the secure cabinet. This includes all documents and electronic data. They will be stored at the Department of Information Science by Professor Holger Regenbrecht. Any personal information held on the participants (names, contact details) will be kept separate. They will also be destroyed at the completion of the research even though the data derived from the research will, in most cases, be kept for much longer or possibly indefinitely. Physical documents will be shredded at the end of the project.  

The results of the project may be published and will be available in the University of Otago Library (Dunedin, New Zealand) but every attempt will be made to preserve anonymity.9.	Disclose and discuss any potential problems and how they will be managed: (For example: medical/legal problems, issues with disclosure, conflict of interest, safety of the researcher, etc)
Generally, we do not anticipate any problems arising out of this experiment for participants. However, there might be some fears about the confidentiality and anonymity of participants. Regarding this, participants will be informed that their participation is voluntary and that all their details will be kept confidential and anonymous following the university policy. No sensitive information will be collected. Each participant will be provided with an information sheet and a consent form to read explaining this before starting the experiment. 
Experiencing virtual reality and wearing a head-mounted display can cause unintended temporary side effects. Rare temporary side effects of experiencing virtual reality can include motion sickness, eye discomfort and fatigue. These side effects are temporary and there are no reported cases of them being permanent. We don’t foresee these side effects occurring; however, we will screen participants if they are predisposed to motion sickness or have had any negative experiences in virtual reality before. If they are predisposed to motion sickness or have had negative experiences using virtual reality in the past, they will be excluded from the study. We have taken steps to mitigate potential motion sickness by using modern hardware (Oculus Rift DK2 HMD) which have low latency and high frame update rate. We also mitigate this potential side effect by requiring participants to only slightly move their head in our application. However, if these rare side effects do occur, the participant will stop the experiment for the moment and take a break. They will be informed that they can stop the experiment if they wish or they can continue if the side effect has alleviated. 
It is possible that the outcome of this project will be used by other reports, publications or conferences. In this situation, no individuals will be identified. 


\chapter{Conclusions}
\chapter{Summary}

%TODO
Acknowledgements 
Bibliography






%---------------Reference commands and structires

% \chapter{Introduction}\label{chapter:introduction}
% \section{Motivation: Teamwork is important. Creative thinking implies generation of ideas, collaboration, and communication. Virtual Reality (VR) is a great enhancement for creative thinking tool-set in architecture.}
%\subsection{Solution validation/evaluation in HCI: methods, and principles.}
\begin{comment}
Methodology: approach to solving the problem; chosen HCI methodology for the final evaluation - no idea
a. Chosen HCI evaluation methodology
\end{comment}


\begin{comment}
See~\autoref{tab:sample}, \autoref{fig:sample-drawing}, \autoref{fig:sample-plot}, \autoref{fig:sample-listing}.
\begin{table}[htpb]
  \caption[Example table]{An example for a simple table.}\label{tab:sample}
  \centering
  \begin{tabular}{l l l l}
    \toprule
      A & B & C & D \\
    \midrule
      1 & 2 & 1 & 2 \\
      2 & 3 & 2 & 3 \\
    \bottomrule
  \end{tabular}
\end{table}

\begin{figure}[htpb]
  \centering
  % This should probably go into a file in figures/
  \begin{tikzpicture}[node distance=3cm]
    \node (R0) {$R_1$};
    \node (R1) [right of=R0] {$R_2$};
    \node (R2) [below of=R1] {$R_4$};
    \node (R3) [below of=R0] {$R_3$};
    \node (R4) [right of=R1] {$R_5$};

    \path[every node]
      (R0) edge (R1)
      (R0) edge (R3)
      (R3) edge (R2)
      (R2) edge (R1)
      (R1) edge (R4);
  \end{tikzpicture}
  \caption[Example drawing]{An example for a simple drawing.}\label{fig:sample-drawing}
\end{figure}

\begin{figure}[htpb]
  \centering

  \pgfplotstableset{col sep=&, row sep=\\}
  % This should probably go into a file in data/
  \pgfplotstableread{
    a & b    \\
    1 & 1000 \\
    2 & 1500 \\
    3 & 1600 \\
  }\exampleA
  \pgfplotstableread{
    a & b    \\
    1 & 1200 \\
    2 & 800 \\
    3 & 1400 \\
  }\exampleB
  % This should probably go into a file in figures/
  \begin{tikzpicture}
    \begin{axis}[
        ymin=0,
        legend style={legend pos=south east},
        grid,
        thick,
        ylabel=Y,
        xlabel=X
      ]
      \addplot table[x=a, y=b]{\exampleA};
      \addlegendentry{Example A};
      \addplot table[x=a, y=b]{\exampleB};
      \addlegendentry{Example B};
    \end{axis}
  \end{tikzpicture}
  \caption[Example plot]{An example for a simple plot.}\label{fig:sample-plot}
\end{figure}

\begin{figure}[htpb]
  \centering
  \begin{tabular}{c}
  \begin{lstlisting}[language=SQL]
    SELECT * FROM tbl WHERE tbl.str = "str"
  \end{lstlisting}
  \end{tabular}
  \caption[Example listing]{An example for a source code listing.}\label{fig:sample-listing}
\end{figure}
\end{comment}