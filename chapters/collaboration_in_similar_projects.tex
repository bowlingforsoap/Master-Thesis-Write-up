\chapter{Related Work}
% TODO: Virtual reality has to be introduced somewhere; and the difference in what I mean by simply VR and immersive VR explained.

This master thesis focuses on two broad topics: collaboration and sound. It puts the work at the intersection of the fields of \gls{vr}, \gls{cscw}, \gls{cve} and \gls{hci}.

\paragraph[]{Collaboration}
Digital collaboration has been addressed in multiple research areas and has a tangled history. Traditionally, it is considered a problem of the \gls{cscw} field, but there was time when it was debated whether it can be considered a distinct research area. \cite{bannon_perspectives_nodate} argues that the first shift towards requirements of a group, instead of an individual has begun when the methods and results provided by field of \gls{hci} started receiving the criticism for their inability to meet the needs of the designers of the collaborative software. The main focus of the \gls{hci} is on providing the proper interface for the interaction between an individual and a computer. The shift of attention to the human-computer-human interaction (HCHI) was caused by the \gls{cmc} movement. Nevertheless, it was mostly a reactive measure, aimed at analyzing how the existing solutions integrate in the actual workspaces, instead of gathering material to help future design and re-design of collaborative systems. The proactive measure came through the establishment of the \gls{cscw} field. The special term "groupware" was coined, which indicated the emphasis on software to support the work of users in groups and their established workflows.

%[CVE - churchil,greenwald]
The next paradigm shift started to happen in the late 1990s, where the already established field of \gls{cscw} started facing the problems of its own, specifically when trying to apply the approach to more open-ended and creative tasks. \cite{churchill_collaborative_1998} argue that the \gls{cscw} approach at the time was better at addressing the routine tasks, such as those with clear-cut requirements for the form of the input and output data: video conferencing, email, co-authoring applications, shared drawing applications, etc. In contrast, the tasks, like design, where users spend significant amount of time discussing and negotiating may suffer from imposing a certain structure early on. In answer to this problem, an approach emerged that has its origins in the domain of \gls{vr} - the \gls{cve}. Churchill and Snowdon describe \gls{cve}s as "distributed, virtual reality that is designed to support collaborative activities". The main difference from the \gls{cscw} was in the freedom of self-expression, interaction with common objects (artifacts) in the scene, and communication. The differences in these approaches can be seen in Fig. \ref{fig:approaches_to_collaboration} and are instantly noticeable: while \gls{cscw} looks like a traditional desktop application, the \gls{cve} reminds more of a computer game, than a tool to promote collaboration during remote collaboration. 

\begin{figure}
	\centering
	\hfill
	\subfloat[CSCW]{\includegraphics[width=0.4\linewidth]{figures/churchill_snowdon_cscw_vs_cve/A-room-created-using-The-Palace-with-two-avatar-embodiments}}
	\hfill
	\subfloat[CVE]{\includegraphics[width=0.4\linewidth]{figures/churchill_snowdon_cscw_vs_cve/Embodiments-in-MASSIVE-1}}
	\hfill
	\caption{Approaches to collaboration (Source: \cite{churchill_collaborative_1998})}
	\label{fig:approaches_to_collaboration}
\end{figure}

%[immersive CVE - greenwald cocoverse]
The next leap in digital collaboration was initiated with the introduction of the first generation \gls{vr} setups from Oculus Rift and HTC Vive. Their sub-millimeter tracking of head and hand-held controllers, high-resolution, high fidelity and wide field-of-view displays eased the problem of foreign interface  requiring high attention load, the initial reason why \gls{cve}s found much broader application in gaming, than in distributed workspaces. More detail on the history of the \gls{cve}s from the introduction to the topic by Churchill and Snowdon to the current day \gls{hmd}s for immersive \gls{vr} can be found in \cite{greenwald_technology_2017}.

\begin{figure}
	\centering
	\includegraphics[width=0.7\linewidth]{figures/Cocoverse}
	\caption{CocoVerse, environment affordances and avatars (Source: \cite{greenwald_cocoverse_nodate})}
	\label{fig:cocoverse}
\end{figure}

\paragraph[Related Work]{}
We will now take a look at the some of the problems that were appraoched in the domain of digital collaboration.

% Interactions and navigation // THis is more a single-player problem
% Greenspace II
Some collaborative projects address the question of interactions with shared artifacts. From the earlier studies, \cite{davidson_greenspace_1996} present an immersive \gls{vr} system with 6 \gls{dof} tracked head and hand movements that facilitates design review and discussion of the architectural design decisions. This was much an exploratory study, where participants got to explore the design of a 3D environment in an interactive fashion. The authors report their findings on the navigation, communication, manipulation and some social affordances of the resulting application.
% Lena
\cite{lena_real-time_nodate} presents a research into what constitutes an interactions in \gls{vr} and utilizes the findings in the \gls{cdp} project to connect two "different realities": an interactive table computer and a CAVE-environment. Intentional non-verbal communication and embodiment are also explored.

% Presence
% CocoVerse
Next popular topic of research is the different types of presence, also described as a feeling of "being there".
Greenwald et al. (\cite{greenwald_cocoverse_nodate}, \cite{greenwald_investigating_2017}, \cite{greenwald_technology_2017}) are exploring the utility of shared immersive \gls{vr} for collaborative learning and co-operation. Authors present a system that is the first of its kind in the research community multi-user framework for collaboration and co-creation, CocoVerse (Fig. \ref{fig:cocoverse}). The topics addressed are: social presence (the quality of the medium in convincing users in the salience of the others, present in the same \gls{ve}), co-presence (psychological and emotional interactions between the users), embodiment (how users are represented), as well as non-verbal communication (i.e. through gestures).
Even though, it is already closer to our initial problem formulation, this stream of research focuses mostly on the phase where users' attention is equally engaged on the same subject.

% Awareness
\cite{gutwin_descriptive_2002} split a collaborative situation into two parts: the domain task and the collaborative part. Ammi and Katz, in their study on communication via audio and haptic feedback in abstract and non-visual environments, prior to implementation, derive requirements to the individual (domain) and collaborative parts of the application \cite{ammi_intermodal_2015}. The former include the concept of \gls{sa}, and the later consists of process and \gls{wa}.
% Situation & Workspace awareness differences
\gls{sa} is defined by Endsley as: "the perception of the elements in the environment within a volume of time and space, the comprehension of their meaning, and the projection of their status in the near future" \cite{endsley_situation_1988}. In-fact, \gls{wa} is a special type of \gls{sa} in shared workspaces. The first difference is that the focus of the \gls{sa} on the domain task only is extended with the focus on collaborative task. Secondly, the extreme flow of information that is typical for \gls{sa} due to its roots in the jet fighter simulations, is reduced in volume to match the use case of shared collaboration between colleagues. Main reason for this is, as authors put it: "sorting slides on a table does not seem very similar to air combat in a jet fighter" \cite{gutwin_descriptive_2002}.

% Similaritites
The part that is similar about the two types of awareness, and that makes this topic interesting in the context of current thesis, is that in both cases users are unable to gather the information they need. If, in case of a jet fighter simulation, the pilot is simply overwhelmed by the sheer volume of the information presented, then in the sorting slides task the users is failing to gather information due to the fact that the system does not present them with adequate awareness information by default.

% Workspace awareness framework: types of communication
The authors go ahead and introduce a descriptive framework to help groupware
designers determine what types of awareness presentations to include in their systems, and allow the comparison of the systems based on the extent to which they promote \gls{wa}.
The framework consists of three parts:
\begin{description}
	\item[Part I] What information makes up \gls{wa}?
	\item[Part II] How is the \gls{wa} information gathered?
	\item[Part III] How is \gls{wa} used in collaboration?
\end{description}

The second part of the framework provides insights about the different sources of \gls{wa} information: 
\begin{enumerate}
	\item Bodies and consequential communication;
	\item Artifacts and feedthrough;
	\item Conversation, gesture, and intentional communication.
\end{enumerate}
While 1 and 3 can be summarized as non-intentional and intentional communication, feedthrough is a mechanism of interest for our purposes, as it implies that artifacts (objects in the \gls{ve}) give off information (audio, visual, etc.) when manipulated, which simultaneously serves as a feedback to the person performing the manipulation and as a notification to the others present in the workspace.
% Gutwin_chalk_2011
The authors utilize this mechanism in their study on the use of synthesized audio to promote to promote \gls{wa} in a shared chalkboard application \cite{gutwin_chalk_2011}. In the systems, users are tasked with tracing a 2D shape (i.e. an outline of a ship), which is accompanied with an auditory feedback. There are two types of users in this environment - the real one (participant) and the simulated (user agent). The later is responsible for drawing shapes in the off-screen part of the workspace, while the former has two tasks - the primary and secondary. The primary task is to trace a given shape, and the secondary is to report that they noticed changes to the environment caused by the user agent. Among other topics, the authors explore the effect different awareness presentations and their combinations (type of auditory feedback, availability of the minimap of the environment, workspace clutter, and attention load) have on the \gls{wa}. They report a substantial increase in awareness due to the use of audio to convey information.
\begin{figure}
	\centering
	\includegraphics[width=0.7\linewidth]{figures/gutwin_chalk_2011}
	\caption{Shared chalkboard application, tracing shape and minimap (Source: \cite{gutwin_chalk_2011})}
	\label{fig:gutwinchalk2011}
\end{figure}

In the next chapter we take a closer look at some of sound properties that can be utilized our application to counter the unexpected changes to the environment and enhance the \gls{wa}.

\begin{comment}


% Gutwin
Gutwin and Greenberg explore the effects of synthesized sound in a shared chalkboard \gls{cscw} on the \gls{wa} ("the up-to-the-moment understanding of another person’s interaction with a shared workspace"). In this system two users are sharing a 2D workspace (a chalkboard). One user was real (participant) and another was simulated (agent). The agent would draw in an off-screen part of the workspace, this activity was accompanied by a synthesized sound. The real user had two tasks - primary and secondary. The former was to trace a given 2D shape (i.e. an outline of a ship) in their part of the shared workspace, and the later was to indicate whether they noticed the changes caused by the agent. % TODO: I don't look so deep into other studies
The authors study the extent of the ability of auditory cues to signal changes in the workspace, in comparison and in combination with a minimap.

% Ammi & Katz
\cite{ammi_intermodal_2015} study the application of abstract and non-visual \gls{vr} environments in the context of improving communication. The system facilitates a collaborative search of targets with the help of audio and haptic feedback. The authors work out a specification for both the individual and the collaborative aspects of the task. The concept of \gls{sa} is used to address the former, and workspace and process awarenesses - the later.

As we have seen, even in the fields of \gls{cscw} and \gls{cve}, a broad range of topics can be addressed with respect to collaboration..
We have viewed non-visual, 2D, immersive \gls{vr} environments 
These are only some..
We will now take a closer look at the awareness approach, as it provides us with..
% Intentional and unintentional communication? Communication with aware person and unaware person - Nah, the difference is in the type of the task that is being conducted. In gutwin the domain task is being explored, while the others explore the collaborative part.

Collaboration can be approached in different ways, depending on the task at hand. More traditional approach is with \gls{cscw} systems, it works well when the task is routine and just requires either information exchange, or an execution of a pre-defined task \cite{churchill_collaborative_1998}. For more creative and non-trivial tasks, which can be obscured by enforcing a certain workflow early-on, an approach using \gls{cve} is more appropriate.

\paragraph{}
% CVEs
One of the origins of \gls{cve}s is the field of \gls{vr}. This is mostly due to the aim  to allow users creative freedom, and reduce the learning curve by providing an intuitive interface. The task of a \gls{cve} is to facilitate communication and collaboration. They allow synchronous and asynchronous task execution, and provide the support for real-time sharing of visual artifacts \cite{churchill_collaborative_1998}.

\subparagraph{}
% Cocoverse, Greenwald - first system of this kind in the research community
A good and concise review of the history of the \gls{cve}s can be found in \cite{greenwald_technology_2017}. The state of art in the area of CVEs was reached by one of the authors, in \cite{greenwald_cocoverse_nodate} and \cite{greenwald_investigating_2017} he presents CocoVerse - a shared immersive \gls{vr} environment for local collaboration and co-creation (Fig. \ref{fig:cocoverse}).

\begin{figure}
	\centering
	\includegraphics[width=0.7\linewidth]{figures/Cocoverse}
	\caption{CocoVerse, environment affordances and avatars (Source: \cite{greenwald_cocoverse_nodate})}
	\label{fig:cocoverse}
\end{figure}

%% What they explored, their focus and discoveries
%%% Goals
Authors' main research direction is communication, collaboration, and learning in \gls{vr}. As the field of immersive \gls{vr} is relatively young, one of the main goals of those projects was to explore the feasibility and utility of such setup, and pilot methodologies for studying behavior in this setting (\cite{greenwald_investigating_2017}). The utility was analyzed by comparing different activities in the 2D viewing mode (via a monitor) and as the immersive 3D experience.

%%% Setup/system
The system is a room-scale shared immersive \gls{vr} experience, which allows users to synchronously manipulate the environment with a set of hand-based tools (create 3D drawings, communicate via rough hand gestures, etc.). CocoVerse also lets the users know where others are looking, by providing them with minimalistic avatars that convey the gaze direction. Users require either a \gls{hmd} and a pair of 6 \gls{dof} controllers to use the system in the immersive mode, or simply a PC to participate in the 2D mode.

%%% Focus and the results
Authors report the overall success and promise of such setups as an engine for learning and creativity. They further review, how the sense of social presence is influenced by the complexity of avatars and high movement realism. \cite{greenwald_investigating_2017} proposes guidelines for deciding on the level of sophistication of the embodiment based on the type of experience that is being digitalized.

% Concise: what other CVE projects addressed.
Among other CVEs that were explored, \cite{lena_real-time_nodate} proposes a system, which connects two different mediums (an interactive table computer, and a CAVE-environment), and focuses on the way to develop consistent interactions across the environment. % TODO: more (e.g. Kulik)

% Conclusion: something else is needed
Generally, it seems that the focus of the modern \gls{cve} projects is more on exploring the aspects of the intentional communication and interactions, where users are always aware of each other during the collaboration task. I our case, the focus should be more on consequential communication, which arises as a result of user's actions.

\subparagraph[CSCW]{}
% CSCWs
We will now turn to the field of \gls{cscw}, which is regarded by \cite{churchill_collaborative_1998} as the main way to approach the automation of the collaborative work, before the paradigm shift towards \gls{cve}s. \gls{cscw} applications also provide distributed access to a shared context. However, unlike \gls{cve}s, these applications provide a less malleable environment that is restricted by requirements of a certain workflow. Examples of \gls{cscw} applications include: emailing software, project management and conferencing tools, shared calendars and drawing tools, etc.

% Chalkboard, Gutwin
A notable work from the field of \gls{cscw} is the study presented in \cite{gutwin_chalk_2011}. The authors study the use of auditory cues to promote the up-to-the-moment understanding of another person’s interaction with the shared chalkboard application, or as they call it, the "Workspace Awareness" (WA) \cite{gutwin_descriptive_2002}. % TODO: I use the same quote in the Awareness chapter. Change.

%% What they explored, their focus and discoveries
%%% Goals
Two main goals of the study were to figure out "How much information can sound convey?" and the effectiveness of audio awareness in a \gls{cscw} application.

%%% Setup/system
The system is a 2D shared drawing application with one real (participant) and one simulated user (called an \textit{agent} in the \gls{cve} terminology). The participant has two tasks, primary and secondary. The primary task was to trace a given 2D shape (Fig. \ref{fig:gutwinchalk2011}), and the secondary was to keep track of what is happening in the shared workspace. When a participant and the agent drew, the system would emit a spatialized sound of chalk writing on a chalkboard (at a lower volume for the participant's actions). An additional way to monitor the shared environment, or as authors call it - a type of awareness presentation, was a minimap (what authors call a \textit{radar}), which showed the complete environment and the local chunk a participant was working in (indicated with a blue rectangle on the minimap).
% The workspace is seen in the figure
% Different awareness presentations - auditory and minimap
% Varying difficulty of the tracing shape

\begin{figure}
	\centering
	\includegraphics[width=0.7\linewidth]{figures/gutwin_chalk_2011}
	\caption{Shared chalkboard application, tracing shape and minimap (Source: \cite{gutwin_chalk_2011})}
	\label{fig:gutwinchalk2011}
\end{figure}


%%% Focus and the results
One of the characteristics of the \gls{wa} is that it is a peripheral (or a secondary) task by nature \cite{gutwin_descriptive_2002}, % TODO: I might be using the same quote in the Awareness chapter. Change.
as such, \cite{gutwin_chalk_2011} attempt to keep participant's attention on the main task by varying the difficulty of the tracing activity: the tracing shape was made to oscilate occasionally. Authors also study the effect of the workspace clutter, the size of the minimap, the type of auditory cues, and the awareness presentation on the \gls{wa}.

The authors report significant improvements to the group awareness in cases, where it is hard to attend to the visual displays, or the line of sight is obscured. Additionally, they provide their thoughts with regards to how and why the audio awareness helps in the collaborative scenarios, as well as the possible limitations of its application.

\paragraph[Bridge]{}
The approach used in \cite{gutwin_chalk_2011} is based on their previous work on the topic of \gls{wa} \cite{gutwin_descriptive_2002}.
Next, we are going to take a closer look at the awareness approach % TODO: did I define it?
, its history, and how it aids the design of the groupware systems. 


\end{comment}





\begin{comment}
% Intro: Introduce what this whole chapter is about

This chapter is going describe the ?status quo of the Collaborative Virtual Environments, highlight the state of art, and the experiment on the Workspace Awareness that the practical part of this thesis took as the base.

\section{Status quo}
.. Like theory part
Define CVE, Groupware, and cooperation levels
% Talk about users and agents, cause I mention it in the experminets chapter.

\section{Related projects}
Mention some projects that are related, but won't be discussed in the detail in this work. The purpose of this subsection is to provide an overview of the current state of the groupware/collaborative software systems.
Lena
® Provides a good overview of the Collaboration and interaction in similar projects
® Focuses more on interactions …
Greenspace II
® A related work from the architectural filed
® Morally aged, Greenwald uses some findings from it
[some other works]

\section{State of art: Multi-User Framework for Collaboration and Co-Creation in Virtual Reality}
§ In this section I will discuss the work I found, which serves as the state of art for current development of Collaborative Virtual Environments (CVEs) - Multi-User Framework for Collaboration and Co-Creation in Virtual Reality (Greenwald et al. 2017).


\section{Parent/precursor/foundation work: The Effects of Dynamic Synthesized Audio on Workspace Awareness in Distributed Groupware}
§ Introduce Gutwin et al. 2011 work on The Effects of Dynamic Synthesized Audio on Workspace Awareness in Distributed Groupware (Chalk Sounds)

Goals

Setup/system
® …
® 

\begin{table}[]
	\begin{tabular}{|l|l|}
		\hline
		Number of people                                                                                  & Conceptually, multiuser                                                                                                \\ \hline
		Medium                                                                                            & PC with an interactive display for drawing                                                                             \\ \hline
		\begin{tabular}[c]{@{}l@{}}Affordances \\ (TODO: make it correspond to WA framework)\end{tabular} & \begin{tabular}[c]{@{}l@{}}Manipulation: 2D drawing, self-report\\ Communication: auditory icons, minimap\end{tabular} \\ \hline
		?Cooperation level                                                                                & Conceptually, 3.2                                                                                                      \\ \hline
		Locality                                                                                          & Shared physical space and remote collaboration                                                                         \\ \hline
	\end{tabular}
\end{table}

Focus and the results
\end{comment}
