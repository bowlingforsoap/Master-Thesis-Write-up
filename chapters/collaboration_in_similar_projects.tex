\chapter{Collaboration in Virtual Environments} %\chapter{Collaboration in Similar Projects}

% Collaborations is a broad topic
% Was approach from different fields: HCI and Computer Supported Cooperative Work (CSCW, also mentioned as Collaborative work in Churchill, 1998)
% Little background for both together, maybe citing Greenberg
% "Greenberg et al. 2017 provide a concise overview of the history of Collaborative Virtual Environments (CVEs) in their work on the applcations of collaboration in learning"

% ?Levels of cooperations - No, cause I don't mention it later anyhow // I sorta do, tho, in the table where I describe different collaboration projects










% Intro: Introduce what this whole chapter is about

This chapter is going describe the ?status quo of the Collaborative Virtual Environments, highlight the state of art, and the experiment on the Workspace Awareness that the practical part of this thesis took as the base.

\section{Status quo}
.. Like theory part
Define CVE, Groupware, and cooperation levels
% Talk about users and agents, cause I mention it in the experminets chapter.

\section{Related projects}
Mention some projects that are related, but won't be discussed in the detail in this work. The purpose of this subsection is to provide an overview of the current state of the groupware/collaborative software systems.
Lena
® Provides a good overview of the Collaboration and interaction in similar projects
® Focuses more on interactions …
Greenspace II
® A related work from the architectural filed
® Morally aged, Greenwald uses some findings from it
[some other works]

\section{State of art: Multi-User Framework for Collaboration and Co-Creation in Virtual Reality}
§ In this section I will discuss the work I found, which serves as the state of art for current development of Collaborative Virtual Environments (CVEs) - Multi-User Framework for Collaboration and Co-Creation in Virtual Reality (Greenwald et al. 2017).
Goals
Setup/system
Focus and the results

\section{Parent/precursor/foundation work: The Effects of Dynamic Synthesized Audio on Workspace Awareness in Distributed Groupware}
§ Introduce Gutwin et al. 2011 work on The Effects of Dynamic Synthesized Audio on Workspace Awareness in Distributed Groupware (Chalk Sounds)

Goals

Setup/system
® …
® 

\begin{table}[]
	\begin{tabular}{|l|l|}
		\hline
		Number of people                                                                                  & Conceptually, multiuser                                                                                                \\ \hline
		Medium                                                                                            & PC with an interactive display for drawing                                                                             \\ \hline
		\begin{tabular}[c]{@{}l@{}}Affordances \\ (TODO: make it correspond to WA framework)\end{tabular} & \begin{tabular}[c]{@{}l@{}}Manipulation: 2D drawing, self-report\\ Communication: auditory icons, minimap\end{tabular} \\ \hline
		?Cooperation level                                                                                & Conceptually, 3.2                                                                                                      \\ \hline
		Locality                                                                                          & Shared physical space and remote collaboration                                                                         \\ \hline
	\end{tabular}
\end{table}

Focus and the results

