% !TeX root = ../main.tex
% Add the above to each chapter to make compiling the PDF easier in some editors.

\chapter{Experiments}
\section{Pilot Studies}
\subsection{Sound Speed and Type}
\subsection{Spatial Judgement}
\section{Workspace Awareness in Immersive Virtual Reality}
% Goal of the study

\paragraph{Methods}

% Participants, Procedure, Task

% Apparatus: Unity, Resonance Audio, Marshall Major II Stereo headphones, Vive

% Study Factors and Conditions: what my factors are, conditions ?= independer variables' values

% Experimental Design
\cite{gutwin_chalk_2011}:
* Awareness presentation was rotated for each participant, so that each presentation was seen in the same position an equal number of times
* There were eight trials in each condition, meaning that there were 24 data points measured per user in each session
* Only one dependent measurement was collected - reaction speed in determining the position of a moving \gls{vb}.

%Gutwin vs our study%
% TODO: add a pic: Gutwin's (vertical) and our (horizontal) working (sound) plane

\begin{table}[]
  \caption{Study comparison}
  \label{table:study_comp}
  \begin{tabular}{|l|l|l|}
  \hline
                             & Gutwin                & This study           \\ \hline
  Primary (distraction) task & Same as the secondary & Contextually related \\ \hline
  Working (sound) plane      & ?Vertical             & Horizonatal          \\ \hline
  Sound type                 & Auditory icons        & Auditory icons       \\ \hline
  Object of analysis         & Workspace awareness   & Workspace Awareness  \\ \hline
  \end{tabular}
\end{table}


\paragraph{Study limitations}
% the fact that we only sample the level 1 SA/WA (we won't be going into sampling direction of translation guesses from the participants)

\paragraph{Results}