% !TeX root = ../main.tex
% Add the above to each chapter to make compiling the PDF easier in some editors.

\chapter{Literature Review}
\section{Information Processing}
\subsection{Information processing loop}
%introduce information processing loop and its components (Perception, Cognition, ...)
%Cues: show examples of how different perceptual cues (including auditory) are registered and processed by humans
\subsection{Auditory Cues}
%introduce auditory icons and earcons, explain the difference, and show the "alarming" role the sound can have 

\section{Awareness} %guide the reader from "giant buildings moving at you" to workspace awareness
\subsection{Situation Awareness}
%introduce SA as an up-to-date understanding of the immediate surroundings
\subsection{Workspace Awareness}
%present WA as being a subfield of SA that is more focused on day-to-day/routine/non-critical work, as opposed to SA that is used in nuclear field, pilot cockpit awareness studies, etc.
\subsection{Related Work: Shared Chulkboard Application}
%review the study by Carl Gutwin et al. that I'm basing my final study on