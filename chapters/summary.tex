% !TeX root = ../main.tex
% Add the above to each chapter to make compiling the PDF easier in some editors.

\chapter{Summary}

%What I studied || This research combines 
This research aimed to address the seeming problem of the surprising events in the shared immersive \gls{vr} environment and their influence on collaboration.

After the initial validation of the problem, the focus of the research was indicated: sonification and collaboration. I then showed that both of them map into the first two stages of the so-called information processing loop, which summarizes how we, as humans, process the information from the environment and our body.

Sonification resides on the perception stage, which is responsible for receiving different perceptual cues, including auditory, and matching them against the our previous knowledge for interpretation.
Next, we step into the cognition territory, where the decision to either save the interpreted information into working memory, or perform an action based on it is taken. 
Two related cognition-level concepts, situation and workspace awareness, were analyzed and their utility in providing information about the current domain and collaboration taks was shown.

At this point the focus of the research become more structured and went from satisfaction with collaboration experience, to the study of the effect additional audio feedback has on \gls{wa} in collaborative immersive \gls{vr} environment.

Two pilot studies and one formal study were conducted. The latter was based on the study by \cite{gutwin_chalk_2011}, but changed the context to the architectural design, and the type of system became an immersive \gls{cve} as opposed to a \gls{cscw} application in the original study. The results support previous findings and show a significant improvement in awareness due to additional audio feedback.

 

%The resulting system

%Results

%Future work


% TODO: system classification (according to the sonification chapter [peripharal to perepheral-serepeious type {novice to veteran users}], to wa framework, etc.)

% ...

% Future work, etc.
\subparagraph[Future work]{}
% the fact that we only sample the level 1 SA/WA (we won't be going into sampling direction of translation guesses from the participants)
% could also try to describe it according to WA framework
% frame drop

\textit{Frame rate} The implementation of the voxel drawing system was not aimed at maximizing performance, as the result the frame rate dropped to around 70 FPS, when completing some of the shapes (i.e. the Low Poly Trees). One participants indicated that they felt a little bit dizzy due to this problem at one point.

\textit{Minimap} Some users indicated that the minimap implementation was not the most convenient, especially in the \textit{Minimap Only}, where to monitor the map they had to take their attention of the main task. While this will most likely be the case in any implementation, a possible solution would be to see the effects of having the minimap attached to the users head in a non-obstructive way.

\textit{Controls} Some users indicated that the controller mappings chosen for the final study were not convenient (Appendix \ref{app:finalstudy_controls}). Specifically, the fact that the main function of the 3D brush controller was mapped to the trigger, but for the minimap controller - to the touchpad. The controls were chosen this way, because drawing with your thumb constantly on the touchpad did not seem so convenient, however, the laser pointer's capabilities map perfectly to the two input affordances for the touchpad, the touch and press, while the pressure sensitivity of the trigger does not provide such a clear separation between the press levels. This ambiguity of choice indicates that physical ergonomics factors should also play a role, when implementing the actual system for collaboration in immersive \gls{vr}.

% TODO: Support the Gutwin findings; his participants also found both awareness displays to be the most useful setup; Also, address this in abstract, too.