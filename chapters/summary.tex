% !TeX root = ../main.tex
% Add the above to each chapter to make compiling the PDF easier in some editors.

\chapter{Summary}

% TODO: system classification (according to the sonification chapter [peripharal to perepheral-serepeious type {novice to veteran users}], to wa framework, etc.)

% ...

% Future work, etc.
\subparagraph[Future work]{}
% the fact that we only sample the level 1 SA/WA (we won't be going into sampling direction of translation guesses from the participants)
% could also try to describe it according to WA framework
% frame drop
\textit{Frame rate} The implementation of the voxel drawing system was not aimed at maximizing performance, as the result the frame rate dropped to around 70 FPS, when completing some of the shapes (i.e. the Low Poly Trees). One participants indicated that they felt a little bit dizzy due to this problem at one point.

\textit{Minimap} Some users indicated that the minimap implementation was not the most convenient, especially in the \textit{Minimap Only}, where to monitor the map they had to take their attention of the main task. While this will most likely be the case in any implementation, a possible solution would be to see the effects of having the minimap attached to the users head in a non-obstructive way.

% TODO: Support the Gutwin findings; his participants also found both awareness displays to be the most useful setup; Also, address this in abstract, too.