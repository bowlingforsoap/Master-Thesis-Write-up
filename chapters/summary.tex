% !TeX root = ../main.tex
% Add the above to each chapter to make compiling the PDF easier in some editors.

\chapter{Summary}

%What I studied || This research combines 
The initial motivation for this research was to conquer the problem of the influence unexpected events might have on the overall satisfaction with the collaboration experience in immersive \gls{vr} environment.
%This research was motivated by the aim to address the seeming problem of unexpected events during collaboration in immersive \gls{vr} environment and their influence on the satisfaction with collaboration.
After the initial validation of the problem, the utility of sonification in monitoring was discovered, and the focus of the research became a combination of collaboration and sonification. I then showed that both topics map into the first two stages of the so-called information processing loop, which summarizes how we, as humans, process the information from the environment and our body.

Sonification was shown to reside on the perception stage, which is responsible for receiving different perceptual cues, including auditory, and matching them against the previous knowledge for interpretation. The cognition stage, where the decision to either save the interpreted information into working memory, or perform an action based on it is taken, was shown to be responsible for awareness during collaboration. 
Two related cognition-level concepts, situation and workspace awareness, were analyzed and their utility in providing information about the current domain and collaboration aspects of the task was shown.

At this point the focus of the research was finalized and went from satisfaction with collaboration experience, to the study of the effect additional audio feedback has on \gls{wa} in collaborative immersive \gls{vr} environment.

Two early prototypes, two pilot studies, and one formal study were conducted. The latter was based on the study by \cite{gutwin_chalk_2011}, but changed the context to the architectural design, and the type of system to an immersive \gls{cve}, as opposed to a \gls{cscw} application in the original study. 
The results support previous findings and show a significant improvement in awareness due to additional audio feedback.
% more detail, go through each individual result mu-fakka:
Initial prototypes indicated that the majority of participants found the sensation of real-sized buildings passing close by or appearing in-front of you in immersive \gls{vr} to be unpleasant. The pilot studies have shown that, in this particular setup, participants preferred auditory icons (or, as they are also called, - representational earcons), as opposed to abstract earcons. Additionally, pilots have shown the inability of stereo headphones to unambiguously convey spatial information. Nevertheless, the observations in the final study indicate that multisensory processing helps correctly interpret the initially ambiguous information in the auditory channel, once the sound source was visually located.
%however, its perceived importance and unexpectedness may go down with time & users didn't report being annoyed or distracted by the sound, however, in longer sessions that might be the case. Therefore, - possibility to turn off the volume, and possibly some accompanying visual cues (Like minimap, which should be improved).

% Go through issues and shortcommings in the same order 
Several questions were not addressed in this research, but play an important role for the final collaboration experience. First, as was previously mentioned in the sonification section, the perceived importance of events happening in the workspace can diminish with time. Already during the initial prototypes, it was apparent that some participants were more affected by the appearing and moving buildings than the other. A longer study would be needed to see how prolonged exposure affects the participants' reaction to unexpected events. 
Annoyance and intrusion is an opposite problem, but may also occur in this setup. Even though, no participants in the final study reported their dissatisfaction with the presence of audio feedback, in the real working environment this might be an issue. Supplying the system with an option to disable the audio feedback would help with both eliminating annoyance on request and studying the effects of exposure to the sudden appearances of the buildings. Additionally, in this research I only tested two different sounds for audio feedback, more sounds with rich "musical" properties should be tested to find the least intrusive and the most fitting to this particular use case.

Visual displays were also attended only scarcely in this research. As a result 

Other directions that have great influence on the satisfaction with the domain and collaboration aspects of the collaboration task, are interactions and representation, respectively. The former is a study of the appropriate interaction mechanisms with the \gls{ve}, depending on the task at hand. The latter, addresses the appearance users have in collaborative environments (also called avatars). A good choice of user avatars and the information they communicate (gaze direction, mimics, gestures, etc.) can have a positive effect on the system (\cite{gutwin_cocoverse_nodate}, \cite{lena_real-time_nodate}), without providing unnecessary distractions (i.e. an uncanny valley situation, where users might get drawn away due to the appearance). The study of representation is closely related to the study of presence, the "sense of being in the \gls{vr}" \cite{schubert_experience_2001}.

In conclusion, collaboration through the digital medium is a vast topic. Many factors have to be taken into account, such as required type of interactions, level of detail of avatars, and awareness. This research studied \gls{wa} during collaboration in immersive \gls{vr} environment and have shown the benefits of audio awareness presentation. The results of the experiments were presented and discussed, and the further research directions were specified.
% collaborative immersive \gls{vr} environments are a young field
% awareness - just one of the factors that have to be taken into account, with interactions, representation, and presence.
% This research has shown that the previous findings from the field of \gls{cscw} are still relevant in the collaborative immersive \gls{vr} environemnts
% 





% TODO: system classification (according to the sonification chapter [peripharal to perepheral-serepeious type {novice to veteran users}], to wa framework, etc.)

% TODO: Support the Gutwin findings; his participants also found both awareness displays to be the most useful setup; Also, address this in abstract, too.