% !TeX root = ../main.tex
% Add the above to each chapter to make compiling the PDF easier in some editors.

\chapter{Summary}

%What I studied || This research combines 
This research aimed to address the seeming problem of the surprising events in the shared immersive \gls{vr} environment and their influence on collaboration.

After the initial validation of the problem, the focus of the research was indicated: sonification and collaboration. I then showed that both of them map into the first two stages of the so-called information processing loop, which summarizes how we, as humans, process the information from the environment and our body.

Sonification resides on the perception stage, which is responsible for receiving different perceptual cues, including auditory, and matching them against the our previous knowledge for interpretation.
Next, we step into the cognition territory, where the decision to either save the interpreted information into working memory, or perform an action based on it is taken. 
Two related cognition-level concepts, situation and workspace awareness, were analyzed and their utility in providing information about the current domain and collaboration taks was shown.

At this point the focus of the research become more structured and went from satisfaction with collaboration experience, to the study of the effect additional audio feedback has on \gls{wa} in collaborative immersive \gls{vr} environment.

Two pilot studies and one formal study were conducted. The latter was based on the study by \cite{gutwin_chalk_2011}, but changed the context to the architectural design, and the type of system became an immersive \gls{cve} as opposed to a \gls{cscw} application in the original study. 
The results support previous findings and show a significant improvement in awareness due to additional audio feedback. % more detail, go through each individual result mu-fakka: 1) sensation was unpleasant, however, its perceived importance and unexpectedness may go down with time & users didn't report being annoyed or distracted by the sound, however, in longer sessions that might be the case. Therefore, - possibility to turn off the volume, and possibly some accompanying visual cue. 2) Like minimap, which should be improved.

 

%The resulting system

%Results

%Future work: Interactions, Avatars
Ergonomics and visualizations
Interactions selection and manipulation
Avatars level of realism





% TODO: system classification (according to the sonification chapter [peripharal to perepheral-serepeious type {novice to veteran users}], to wa framework, etc.)

% ...

% Future work, etc.
\subparagraph[Future work]{}
% the fact that we only sample the level 1 SA/WA (we won't be going into sampling direction of translation guesses from the participants)
% could also try to describe it according to WA framework
% frame drop

% TODO: Support the Gutwin findings; his participants also found both awareness displays to be the most useful setup; Also, address this in abstract, too.